\begin{enumerate}

\item  Sketch the region below this line, above the $t$-axis, and between the vertical lines $t=1$ and $t = 4$.

\smallskip

\item  Use geometry to find the area of the region.

\smallskip

\item  Now sketch the region below the line $y=2t -2$, above the $t$-axis, and between the lines $t=1$ and $t=x$ for some $x > 1$.

\smallskip

\item  Use geometry to find the area of this region as a function of $x$.  Call this area, your function, $A(x)$.

\smallskip

\item  Take the derivative of the area function $A(x)$.

\smallskip

\end{enumerate}

\begin{enumerate}

\item  For some $x>1$, sketch the region that the function $\displaystyle{A(x) = \int_{-1}^x (3 + t^2)\,dt}$ represents the area of.

\item  Use the fact that $\displaystyle{\int_a^b \frac{b^3-a^3}{3}}$ and $\displaystyle{\int_a^bc\,du = c(b-a)}$, and the rules of combining definite integrals to find an expression for $A(x)$ and simplify that expression.

\smallskip

\item  Compute $A'(x)$.

\smallskip

\item  For a small positive number $h$, sketch the region whose area is represented by \\$A(x + h) - A(x)$.

\smallskip

\item  Use your picture, and maybe a rectangle, to explain why $\displaystyle{\frac{A(x+h) - A(x)}{h} \approx 3 + x^2}$.

\smallskip

\item  Based on part (e), give both an intuitive reason and a logical reason using the limit definition of the derivative for why your answer in (c) makes sense.

\smallskip

\end{enumerate}